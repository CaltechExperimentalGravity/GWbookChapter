\chapter{Indexing\label{ch3}}

\section{Indexing}\index{index}
The first step in producing the index is to put the necessary
\verb|\index| commands in your document. The following example shows
some simple \verb|\index| commands and the index entries that they
produce.

\begin{table}[h]
\begin{tabular}{ll|l}
Page ii:   & \verb|\index{Alpha}              | &  Alpha, ii           \\
Page viii: & \verb|\index{alpha}              | &  alpha, viii, ix, 22 \\
Page ix:   & \verb|\index{alpha}              | &  bites               \\
Page 7:    & \verb|\index{gnat!size of}       | &  \sitem animal       \\
Page 8:    & \verb|\index{bites!animal!gnats} | &  \ssitem gnats, 8, 10\\
Page 10:   & \verb|\index{bites!animal!gnats} | &  \ssitem gnus, 10    \\
Page 10:   & \verb|\index{bites!animal!gnus}  | &  \sitem vegetable, 12\\
Page 12:   & \verb|\index{bites!vegetable}    | &  gnat, 32            \\
Page 22:   & \verb|\index{alpha}              | &  \sitem anatomy, 35  \\
Page 32:   & \verb|\index{gnat}               | &  \sitem size of, 7   \\
Page 35:   & \verb|\index{gnat!anatomy}       | &  gnus                \\
           & \verb|\index{gnus!good}          | &  \sitem bad, 38      \\
Page 38:   & \verb|\index{gnus!bad}           | &  \sitem good, 35
\end{tabular}
\end{table}

You then run \LaTeX\ on your entire document, causing it to generate
the file {\tt ws-book975x65.idx}. Next, run the \verb|MakeIndex|
program by typing the command, \verb|makeindex ws-book975x65|. This
produces the file \verb|ws-book975x65.ind|. If \verb|MakeIndex|
generated no error messages, you can now rerun \LaTeX\ on your
document and the index will appear.

Reading the index, you may discover mistakes, which
should be corrected by changing the appropriate \verb|\index|
commands in the document and regenerating the {\tt ind} file.  If
there are problems that cannot be corrected in this way, you can
edit the {\tt ind} file directly.  However, such editing is
to be avoided because it must be repeated every time you generate a
new version of the index.

If you would like to have ``general'' and ``author'' indexes
separately, please see \sref{sec3.1}. For more details on index
generation, see \cite{lamp87}.

\section{Multiple Indexes}\label{sec3.1}\index{index!multiple indexes}

To create a ``general'' and an ``author'' index, include the following lines:

\begin{verbatim}
...
\usepackage{ws-multind}
\makeindex{general}
\makeindex{authors}
...
\index{authors}{Fairbairns, R.} %\aindx{Fairbairns, R.}
...
\index{general}{FAQs}           %\gindx{FAQ}
...
\printindex{general}{GENERAL INDEX}
\printindex{authors}{AUTHOR INDEX}
\end{verbatim}

To complete the job, compile your file as follows:

\begin{enumerate}
\item latex ws-book975x65
\item latex ws-book975x65
\item bibtex ws-book975x65\qquad(for bibtex users)
\item makeindex general
\item makeindex authors
\item latex ws-book975x65
\item latex ws-book975x65
\end{enumerate}
