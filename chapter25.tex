\chapter[Diagnostics, Control Systems, and People]{Diagnostics, Control Systems, and People\label{ch7}}

\section{Introduction}\label{sec7.1}
The commissioning process has taught us that we were mistaken in the design process.

We could divide things into subsystems as done in the chapters of a book, but then 
to make them work together things do not factor nearly so neatly. Controllability is 
much more important than the last few percent of isolation or noise performance. It
is not wise to design for a single point in parameter space: if each part of the interferometer
can only work when all other pieces are performing as designed, it becomes impossible
to integrate. Its the inverse of the classic locked-room mystery stories.


\section{Detector Diagnostics}
   - we need testpoints everywhere
   - we need to have high frequency noise measurements to diagnose laser noise, HOM resonances, and mirror mode ringups
   - even more important for parametric instability control\cite{Matt:PI}
   - missing diagnostics: non-uniform optic absorption, time dependence of scattering, how close are the optics to the stops, THD of the force actuators, THD of the displacement sensors, strain release in vacuum system couples to IFO by scattered light


\section{LIGO Mirror Suspensions: A Design Example}

Important considerations in test mass suspension design\cite{SUS:2012, Aston:2012}:
\begin{itemize}
   \item Vibration isolation: nearly all of the seismic isolation in the GW detection band ($f > 5\,\rm Hz$)
     comes from the suspension. To fulfill this requirement, the mirror must be isolated by a 
     multi-stage pendulum~\cite{Beker:2011}.
    \item low suspension thermal noise and minimization of creep noise~\cite{Levin:2012ek, Gretarsson:2005gs}
    \item third priority is some thoughts on damping
    \item 4th priority is minimize damping noise
    \item should instead consider upconversion, control forces, gradual noise reduction, minimization of angular noise generation, etc. all together in holistic metric v. sensitivity and uptime.
    \item Future suspension will look different than the LIGO/Virgo style
\end{itemize}

​\section{Control System Synthesis}
  1) LIGO/Virgo/GEO done by pre-1980 controls technology: pole/zero placement, SISO error signal minimization, by-eye estimates of loop stability
  2) Techniques for process and aircraft control use more global and MIMO synthesis techniques with general cost function minimization.
  3) Machine learning techniques can use nonlinear cost functions and account for rare, large excursions -> maximize uptime

In recent years, more sophisticated algorithms have been introduced to better control the vibration isolation
systems supporting the mirrors in the interferometer\cite{Beker:2014, Driggers:2012fl, Ryan:FFW2012}

  4) Adaptive machines can explore non-intuitive noise cancelation techniques​
  5) The dynamical response of the interferometer is a combination of mechanical, opto-mechanical, and electronic feedback. Its a mistake to treat these on such unequal footings - the interferometer design ought to be done using global search methods (such as MCMC) in the same way that we do for BHBH parameter estimation.

\section{Most Surprising Noise Sources}
\begin{itemize}
\item Classic bilinear: beam spot motion makes upconversion. Well understood in the 90's.
\item Limiting noise in S1: seen, but not understood until S2.
\item Weak LO: limit in S3. Found by noticing harmonics in spectra. Same structure as S1 OL noise.
\item Multiple PDs: the signal is not the same.
\item RF AS PD: What's in the I-phase?

\end{itemize}

\section{Risk Register and Risk Reduction Research}
\begin{itemize}
\item Keep a list of risks.
\item Gather detector experts to update list periodically.
\item Rate risks: probability of occurrence, potential impact, cost to fix
\item Risk Breakdown Structure and Risk Management Tools
\end{itemize}

\begin{figure}[h]
\centering
\includegraphics[width=\columnwidth]{Figures/RIPBubbleChartNoOverlays.png}
\caption{Example of graphical representation of a Risk Registry}
\label{fig:RiskBubbles}
\end{figure}


\section{People}
\begin{enumerate}
\item An important piece of the noise reduction feedback system is the team of humans 
  in the control room.
\item Viewed as a NN or ML algorithm, the machine needs to be rewarded, punished, trained, and 
  maintained in order to minimize the time needed to reach the quantum sensitivity limits.
\item The team must also be able to train the next generation of machines.
\item In the Navy they teach the priority list of (1) complete the mission, 
  (2) safety, (3) train your replacement. How should we do this better?
\item How to balance making progress quickly and training the next wave of young scientists?
\item The team needs to make progress quickly, but the PhD students need theses and the local
  observatory staff need to be fully engaged (not steamrolled) by visiting scientists working 15 hour days.
\end{enumerate}

\begin{figure}[h]
\centering
\includegraphics[width=\columnwidth]{Figures/GroupPhoto_LLOworkshop13.jpg}
\caption{International GW Commissioning workshop}
\label{fig:workshopPhotoLLO}
\end{figure}

\section{Conclusion}
These wise things should be kept in mind while doing the detector commissioning.

In addition to the usual thoughts about how



