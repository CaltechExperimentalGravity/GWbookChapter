\chapter[Integrated Detector Commissioning]{Integrated Detector Commissioning}
\label{IDC}

% editing Tex in Atom
% http://rolflekang.com/writing-latex-in-atom/

Feynman: "all things are made of atoms..."

% If, in some cataclysm, all of scientific knowledge were to be destroyed, and only one sentence passed on to the next generation of creatures, what statement would contain the most information in the fewest words? I believe it is the atomic hypothesis that all things are made of atoms — little particles that move around in perpetual motion, attracting each other when they are a little distance apart, but repelling upon being squeezed into one another. In that one sentence, you will see, there is an enormous amount of information about the world, if just a little imagination and thinking are applied.

\section{Introduction}
\label{sec7.1}
If civilization were to be wiped out and only a single statement could be left
carved in stone for future scientists to use, as they rebuilt civilization and
invented gravitational wave detectors to reach out into the universe,
it would be this:\\

"Think carefully and holistically about your detector before building it
such that no part is neglected as being too small or 'second order', yea,
for even the mighty may be brought low by the loose baffle or the
unforseen cross-coupling."

The interferometer commissioning process has taught us that we could improve our design process.\\

We might divide things into subsystems as done in the chapters of a book,
but then to make them work together things do not factor nearly so neatly.
Controllability is much more important than the last few percent of
isolation or noise performance.\\

It is not wise to design for a single point in parameter space: if each
part of the interferometer can only work when all other pieces are
performing as designed, it becomes impossible to integrate.\\

Its the inverse of a Sherlock Holmes locked-room mystery story.


\section{Sub-system Integration Issues}
We design the individual components of the interfererometer to fulfill some
primary goals with some secondary considerations.

But in some cases, the most important driver in the design should, in fact, be the
interaction that this 'isolated' component has with the full system. In some cases,
this interaction is only apparent with the highest circulating laser power levels
and only when considering the smallest mirror displacements.


\begin{figure}[h]
\centering
\includegraphics[width=\columnwidth]{Figures/SystemConflicts.pdf}
\caption{Examples of interactions between subsytems}
\label{fig:SystemConflicts}
\end{figure}

Figure~\ref{fig:SystemConflicts} shows diagrammatically many of the most
significant subsystem conflicts/intercations.

\subsection{Issues due to high laser power}
To increase the phase (and thereby strain) sensitivity of the interferometer
the circulating laser power in the interferometer is increased (into the MW
regime). This has several problematic consequences:
\begin{itemize}
\item Quantum fluctuations of the radiation pressure increase as P$^{1/2}$, increasing the low frequency strain noise.
\item The classical radiation pressure force due to the high cavity power leads to static and dynamic instabilities in mirror orientations~\cite{Sidles:2006un, Dooley:13, aLIGO:ASC}. In addition to producing more mirror motion, the concomitant increase in the angular feedback control gain, injects noise into the GW strain channel at low frequencies (cf.~Chapter~\ref{ASC}), masking the presence of intermediate mass black holes.
\item In addition to angular instabilites, the system of compound resonant cavities can exhibit instabilities along the optical cavity axis~\cite{SGLMW2004, BuCh2002, Osamu:spring}. This
unstable opto-mechanical eigenfrequency can show up in the GW detection band,
making the control system challenging to design (cf.~Chapter~\ref{LSC}).
\item Through the finite optical absorption in the mirror substrates and the
high reflectivity mirror coatings, heat is deposited in the mirrors and thermal
gradients are formed. These lead to thermal lensing and thermal deformations
(cf.~Chapter~\ref{TCS}) which can reduce the stored power in the system, destabilize
the MIMO angular control system, increase the coupling of laser ampltidue and
frequency noise to the strain channel, spoil the dark fringe contrast, and
aggravate opto-acoustic parametric instabilities (cf.~Chapter~\ref{SYS}).
\item Light which scatters out of the main interferometer and then returns after interacting with a vibrating piece of the environment, will be phase modulated by the vibration. In a low power interferometer, this will just produce phase noise, but depending upon the relative phase of the interferometer field and the backscattered light, this can also produce amplitude noise. These amplitude fluctuations (via radiation pressure) can produce as much as an order of magnitude more strain noise than mere phase fluctuations.

\end{itemize}

\subsection{LIGO Mirror Suspensions: A Design Example}
The way the vibration isolation and mirror suspension systems are designed,
provides much attenuation of noise at higher frequencies and some amplification
at lower frequencies (cf.~\ref{fig:SeismicTFs}). This condition is analogous to the
one encountered in analog, electronic filter design;  a sharp transition from
the passband to the stopband requires complex poles in the passband (e.g.
in Chebyshev filters) which amplify the signals at those frequencies. This
can ameilorated somewhat through the use of 'cold damping'
techniques~\cite{Kuroda:1982vf, Forward:1979ks}.


\begin{figure}[h]
\centering
\includegraphics[width=\columnwidth]{Figures/SeismicIsolations.pdf}
\caption{Vibration isolation for the initial LIGO~\cite{ponslet:432, Giaime:1996},
                   Virgo~\cite{Stefano:2001, Virgo:SA2010, Accadia:2011jh},
                   TAMA (with SAS)~\cite{Szabi:TAMASAS},
                   GEO600~\cite{Hartmut:PhD, Ken:GEOseismic, plissi:3055},
                   Adv. LIGO~\cite{aLIGO:Seismic2002},
                    KAGRA~\cite{Somiya:2011tb}, and the Einstein Telescope~\cite{ET2011}. In the KAGRA case,
                    the mechanical links for cooling (included) are expected to limit the
                    isolation performance above $\sim$\,1\,Hz~\cite{Takahashi:email}}
\label{fig:SeismicTFs}
\end{figure}


Important considerations in test mass suspension design\cite{SUS:2012, Aston:2012}:
\begin{itemize}
   \item Vibration isolation: nearly all of the seismic isolation in the GW detection band ($f > 5\,\rm Hz$)
     comes from the suspension. To fulfill this requirement, the mirror must be isolated by a
     multi-stage pendulum~\cite{Beker:2011}.
    \item low suspension thermal noise and minimization of creep noise~\cite{Levin:2012ek, Gretarsson:2005gs}
    \item third priority is some thoughts on damping
    \item 4th priority is minimize damping noise
    \item should instead consider upconversion, control forces, gradual noise reduction, minimization of angular noise generation, etc. all together in holistic metric v. sensitivity and uptime.
    \item Future suspension will look different than the LIGO/Virgo style
\end{itemize}

During the commissioning of the first generation of kilometer scale detectors, it became clear that the angular motion of the suspended optics was too large and a significant challenge for the angular control systems~\ref{ASC}. The most obvious mechanism to produce rotations of the optics is, perhaps, direct coupling of the ground motion through the suspension system and into mechanical torque.

In fact, the dominant mechanism is through the control system: rotations and translations of the ground produce relative motions of the mirrors along the direction of the laser beam propagation. The Length Sensing and Control~\ref{LSC} system then corrects for this by applying forces to the mirror suspension system.

The length to angle coupling has a complicated transfer function from each length actuator to the mirror angle due to the many resonances in the suspension, the presence of the local damping loops, and the Sidles-Sigg effect~\cite{Sidles:2006un, Hirose:10, Dooley:13}.

\subsubsection{High Q Eigenmodes}
In the effort to make a mirror suspension with very low thermal noise, the mechanical
elements nearest the mirror must have a very high mechanical Q
(cf.~Chapters~\ref{THN} and \ref{SUS}).
In the ideal steady-state, these modes are driven only by the thermal energy
dictated by the Equipartition theorem and their presence in the data can be simply
removed through standard techniques~\cite{Allen:1999wy, Finn:Violins, Searle:2003ib, Sintes:1998gq}.

Nature, however, is not so kind. Large seismic transients and instabilites in the control systems can lead to large excitations of these modes. The vertical and roll (rotation about the cavity axis)
modes of the mirror suspension fall in the lower end of the observation band
(f $\simeq$ 10-20 Hz) and the higher frequency fiber modes which have nodes at
their endpoints are at multiples of $\sim$\,500\,Hz. The lower frequency modes have
decay time constants of order several hours, whereas the higher frequency ones
have time constants of several days.

Each of the four test masses has a vertical mode, a roll mode, and 2n fiber modes
for each of its four fibers. So the total number of high Q suspension modes is:
\begin{equation}
N_{high} = 4 \times (1 + 1 + 4 \times 2n)
\end{equation}
which totals to 104, if we consider the first few harmonics only.

In order to allow the interferometer to operate, numerous digital feedback
loops have been designed to damp these modes~\cite{martynov2015lock}. In the
installed systems, the only way to damp these modes is to sense their perturbation
of the mirror motion and apply feedback forces to the suspension chain. In the case
that thermall induced drifts cause the mode frequencies to overlap, the problem
becomes intractable results in massive downtime of the detectors. Future designs
may utilize dedicated hardware solutions to apply damping forces without degrading
the thermal noise~\cite{Dmitriev:2010hk, Lockerbie:2011zma}.

%\subsection{Digital Control Systems: A second example}

% ​\section{Control System Synthesis}
%   1) LIGO/Virgo/GEO done by pre-1980 controls technology: pole/zero placement, SISO error signal minimization, by-eye estimates of loop stability
%   2) Techniques for process and aircraft control use more global and MIMO synthesis techniques with general cost function minimization.
%   3) Machine learning techniques can use nonlinear cost functions and account for rare, large excursions -> maximize uptime

% In recent years, more sophisticated algorithms have been introduced to
% better control the vibration isolation
% systems supporting the mirrors in the
% interferometer~\cite{Beker:2014, Driggers:2012fl, Ryan:FFW2012}

%   4) Adaptive machines can explore non-intuitive noise cancelation techniques​
%   5) The dynamical response of the interferometer is a combination of mechanical, opto-mechanical, and electronic feedback. Its a mistake to treat these on such unequal footings - the interferometer design ought to be done using global search methods (such as MCMC) in the same way that we do for BHBH parameter estimation.






\section{Most Surprising Noise Sources}

\begin{itemize}
  \item Classic bilinear: beam spot motion makes upconversion. Well understood in the 90's.
  \item Limiting noise in S1: seen, but not understood until S2.
  \item Weak LO: limit in S3. Found by noticing harmonics in spectra. Same structure as S1 OL noise.
  \item Multiple PDs: the signal is not the same.
  \item RF AS PD: What's in the I-phase?
\end{itemize}


\subsection{Scattered Light}
The phenomenon of excess phase noise due to scattered light has been known about for
decades~\cite{Schilling:1981} at least.

Due to imperfections in the optics, for example, a small fraction of the laser light escapes from
the main interferometer beam path. This light then interacts with something (e.g. a piece of the
vacuum system or some other suspended optic or beam baffle) and then recombines with the
circulating laser field within the
interferometer~\cite{Kip:Scatter95, Kip:scatter1989, Sam:Scatter2012, Stefano:Scatter, Vinet:scatter, fritschel1998high}.
The recombination may occur at virtually any point within the system: inside of the
Fabry-Perot arm cavities, at the beamsplitter, or even at the final photodetector which
records the GW strain signal.

It is instructive to consider quantitatively the ampltidue of the noise for a few of
these cases. A significant source of scatter for all of the long interferometers
\begin{align}
S_x = some scattering stuff
\end{align}


\section{Risk Register and Risk Reduction Research}

\begin{itemize}
  \item Keep a list of risks.
  \item Gather detector experts to update list periodically.
  \item Rate risks: probability of occurrence, potential impact, cost to fix
  \item Risk Breakdown Structure and Risk Management Tools
\end{itemize}

\begin{figure}[h]
\centering
\includegraphics[width=\columnwidth]{Figures/Risk.pdf}
\caption{Risk Register for Advanced LIGO. Area of each circle represents the impact on the binary neutron star inspiral range.}
\label{fig:RiskBubbles}
\end{figure}

\begin{table}
\tiny
\begin{tabular}{lllrrrrrr}
\toprule
{} &                     Risk Description & sub-System &  Prob. (\%) &  Cost (M\$) &  Time (m) &  BNS (\% Range) &  BBH (\% Range) &  Total Severity \\
%\midrule
0  &            Gravity Gradient too high &        SEI &               75 &              0.30 &                5 &                    10 &                    90 &             NaN \\
1  &  Thermal Distortion destabilizes ASC &        AOS &               60 &              0.50 &               18 &                    30 &                    50 &             NaN \\
2  &               OMC PZT noise too high &        LSC &               15 &              0.10 &                2 &                     5 &                     5 &             NaN \\
3  &               LSC Aux noise too high &        LSC &               25 &              0.03 &                3 &                    15 &                    15 &             NaN \\
4  &          ASC feedback noise too high &        ASC &               65 &              0.03 &                6 &                     5 &                    90 &             NaN \\
5  &                 Beamtube Backscatter &        SYS &               33 &              0.15 &                5 &                     3 &                    65 &             NaN \\
6  &          SRC Mode Healing $\rightarrow$ Harming &        SYS &               50 &              0.50 &                6 &                    40 &                     5 &             NaN \\
7  &       BS clipping $\rightarrow$ contrast defect &        COC &               90 &              0.35 &                6 &                    33 &                     3 &             NaN \\
8  &              ETM Charge fluctuations &        SUS &               50 &              0.20 &                4 &                    15 &                    40 &             NaN \\
9  &             Parametric Instabilities &        LSC &               70 &              0.20 &                4 &                    40 &                     5 &             NaN \\
10 &                   Suspension crackle &        SUS &               25 &              0.25 &                7 &                     5 &                    50 &             NaN \\
11 &        Coating thermal noise anomaly &        COC &               20 &              0.75 &                8 &                    30 &                     3 &             NaN \\
12 &                      OMC backscatter &        LSC &               55 &              0.10 &                3 &                    10 &                     2 &             NaN \\
13 &                       HAM6 Acoustics &        SYS &               70 &              0.10 &                2 &                    10 &                     3 &             NaN \\
14 &                         Residual Gas &        SYS &               40 &              0.60 &                4 &                    30 &                     5 &             NaN \\
15 &                    Bounce/Roll modes &        SUS &               80 &              0.15 &                3 &                    20 &                    95 &             NaN \\
16 &                         Violin Modes &        SUS &               35 &              0.15 &                3 &                    80 &                    10 &             NaN \\
17 &             CDS/SEI breaks SUS fiber &        SEI &               10 &              0.03 &                3 &                   100 &                   100 &             NaN \\
%\bottomrule
\end{tabular}
\end{table}

\section{Detector Diagnostics}
   - we need testpoints everywhere

\begin{itemize}
  \item high frequency noise measurements to diagnose laser noise
  \item HOM resonances
  \item mirror mode ringups
  \item even more important for parametric instability control\cite{Matt:PI}
  \item non-uniform optic absorption
  \item time dependence of optical scattering
  \item how close are the optics to the stops
  \item THD of the force actuators
  \item THD of the displacement sensors
  \item strain release in vacuum system couples to IFO by scattered light
  \item bursts of gas lead to transient phase noise in arm cavities
\end{itemize}

\section{People}
\begin{enumerate}
\item An important piece of the noise reduction feedback system
  is the team of humans in the control room.
\item Viewed as a NN or ML algorithm, the machine needs to be
  rewarded, punished, trained, and maintained in order to minimize
  the time needed to reach the quantum sensitivity limits.
\item The team must also be able to train the next generation of machines.
\item In the Navy they teach the priority list of (1) complete the mission,
  (2) safety, (3) train your replacement. How should we do this better?
\item How to balance making progress quickly and training the next
  wave of young scientists?
\item The team needs to make progress quickly, but the PhD
  students need theses and the local observatory staff need to
  be fully engaged (not steamrolled) by visiting scientists
  working 15 hour days.
\item Include some data here on grad students who did work on the
  interferometers (at CIT and MIT) and what they did next and where
  they are now. e.g., they stayed in the field N years, the left or not.
\end{enumerate}

\begin{figure}[h]
\centering
\includegraphics[width=\columnwidth]{Figures/GroupPhoto_LLOworkshop13.jpg}
\caption{International GW Commissioning workshop}
\label{fig:workshopPhotoLLO}
\end{figure}

\section{Conclusion}
These wise things should be kept in mind while doing the detector commissioning.

In addition to the usual thoughts about how to radically improve the detector sensitivity.
