\begin{preface}
The primary motivation for adaptive memory programming, therefore, is
to group and unify all these emerging optimization techniques for
enhancement of the computational capabilities that they offer to
combinatorial problems which are encountered in real life in the area
of production planning and control.

We confront this pitfall technically, by introducing explicit remarks
about the generality of results at appropriate places;
methodologically, by accumulating enough applications for every major
idea to make its validity and generality stand out; and
philosophically, observing that physics moves forward most of its
ideas by analogies to cleverly chosen simple systems for which
profound intuitions have been formed.

Special attention has also been paid to the wave-current interaction
problems. Several models are evaluated by comparing the numerical
results with laboratory data. It is quite clear that these
higher-order modified equations are adequate for modeling the wave
propagation from deep water to shallow water. However, to apply these
models in the surf zone further study of breaking waves and the proper
parameterization of wave breaking processes are essential.

One of most surprising findings is coastal engineering research in the
last two decades is the robustness of the shallow-water equation
models in calculating the wave runup in swash zone. Although the wave
breaking process is usually not considered in the shallow-water
equations, with a proper tuning of the numerical dissipation as well
as the bottom friction, these models can predict the time history of
runup heights for various types of incident waves with impressive
accuracy. These models have also been extended to examine the
interactions between water waves and coastal structures, which are
either impermeable or protected by a layer.

The second order wave theory must be employed to include the effects
of wave drift forces and springing since they are caused by the
quadratic nonlinearity. To consider ringing and wave slamming, cubic
nonlinearity and higher order nonlinearity must be included in the
formulation.

The factors which have contributed most towards this are the growth of
the uranium industry, the acceptance of solvent extraction as a
process suitable for industrial use on a large scale, the development
of techniques of leaching and reduction at temperatures up to
240$^\circ$C at moderate pressures, and the demand for numerous
less-common metals and other elements.

To model wave slamming and ringing as mentioned above and other
nonlinear phenomenon, it is necessary to undertake fully nonlinear
transient analyses, usually involving numerical time marching. At
present many such numerical models exist. One of common difficulties
faced by these models is the procedure to track the location of free
surface, especially in the case of wave breaking. Different
applications associated with each method, especially in wave
hydrodynamics are discussed. More than one hundred references are
cited in the paper.

\begin{flushright}
{\it T. M. Chan}
\end{flushright}
\end{preface}
